\documentclass[a4paper, 11pt]{article}
\usepackage[a4paper, left=3cm, bottom=4cm, right=3cm]{geometry}

\RequirePackage[english]{babel}
\RequirePackage[utf8]{inputenc}
\RequirePackage{amsmath}
\RequirePackage{amsfonts}
\usepackage{mathtools}
\RequirePackage{hyperref}
\usepackage{tikz}
\usepackage{braket}

\newcommand{\dd}{\mathop{\mathrm{d}\!}{}}
\newcommand{\Tr}{\mathop{\mathrm{Tr}\!}{}}
\newcommand{\deriv}[2]{\dfrac{\dd #1}{\dd #2}}
\newcommand{\pderiv}[2]{\dfrac{\partial #1}{\partial #2}}
\newcommand{\HH}{\mathcal{H}}

\renewcommand\bra[1]{{\langle{#1}|}}
\makeatletter
\renewcommand\ket[1]{%
	\@ifnextchar\bra{\k@t{#1}\!}{\k@t{#1}}%
}
\newcommand\k@t[1]{{|{#1}\rangle}}
\makeatother

\bibliographystyle{alpha}

\date{\today}
\author{Bruno Bucciotti}
\title{Quantum information 2}


\begin{document}
	\maketitle
	
	\begin{abstract}
		We restrict our attention to finite dimensional systems.
		
		\href{https://www.youtube.com/watch?v=GyKcdbFGeV8}{90's}
		
	\end{abstract}

	\tableofcontents
	\clearpage
	\section{Lecture 1}
	Consider a system $S$ coupled to an environment $E$ such that $S+E$ is a closed system described by the hamiltonian $H_{SE} = H_S + H_E + V_{SE}$, where $H_S,H_E$ stand for $H_S\otimes 1_E$, $1_S\otimes H_E$, and $V_{SE}$ represents an interaction term.
	
	States of $S$ are density matrices $\rho_S\in\sigma(\HH_S)$.
	The time evolution operator is $U_{SE}(t) = \exp^{-i H_{SE} t}$. It maps $\rho_{SE}(0)$ into $\rho_{SE}(t) = U_{SE}(t) \rho_{SE}(0) U^\dagger_{SE}(t)$. Assuming the initial state factorizes into $\rho_S(0)\otimes\tau_E(0)$, we get
	\[ \rho_S(0)\rightarrow \rho_S(t) = \Phi\left[\rho_S(0)\right] = \Tr_E\left[ U_{SE}(t) \left(\rho_S(0)\otimes\tau_E(0)\right) U^\dagger_{SE}(t) \right] \]
	The factorization assumption may seem very restrictive, but actually reflects what happens in the lab, where your state preparation procedure outputs a state for your system which is uncorrelated with the environment, over which we have no control.
	\vspace{5mm}
	
	We proceed to study all such \emph{quantum channels} $\Phi$, holding $t$ and $\tau_E(0)$ fixed.
	
	\noindent First, notice that $\Phi$ is a \emph{superoperator}: it acts a linear transformation on operators on $S$.
	
	\noindent Second, $\Phi$ is defined on \emph{all} linear operators $\mathcal{L}(\HH)$, not just $\sigma(\HH)$; in particular, observables. The extension is trivial
	\[ \Phi\left[\hat{\Theta}\right] = \Tr_E\left[ U_{SE}(t) \left(\hat{\Theta}\otimes\tau_E(0)\right) U^\dagger_{SE}(t) \right] \]
	and it is unique if we require linearity (\emph{proof}: any operator can be decomposed as a linear sum of density matrices). Unicity hold even for infinite dimensional systems.
	
	\subsection{Representations of quantum channels}
	\subsubsection{Physical representation}
	The one we just described.
	\subsubsection{Stinespring representation}
	We know we can purify a state by adding extra degrees of freedom. Suppose we purify the state of the environment $\tau_E(0)$ by adding a system $E_1$, $E' = E+E_1$. Then $\tau_E(0) = \Tr_{E_1} \left(\ket{0}_{E'}\bra{0}\right) $. Then, defining $U'_{SE'} = U_{SE} \otimes 1_{E'}$, we get (simple check)
	\[ \Phi\left[\hat{\Theta}\right] = \Tr_{E'} \left[ U'_{SE'} \left( \hat{\Theta} \otimes \ket{0}_{E'}\bra{0} \right) {U'}^\dagger_{SE'} \right] \]
	
	We have thus proved that the stinespring representation is not only a special case of physical representation, but equivalent to it. They are both \emph{extrinsic} representations.
	
	\subsubsection{Kraus representation}
	Kraus representation is \emph{intrinsic}, no environment needed. It is defined in terms of a Kraus set $M_k\in \mathcal{L}(\HH_S)$.
	\[ \Phi\left[\hat{\Theta}\right] = \sum_{k=1}^D M_k \hat{\Theta} M_k^\dagger,\qquad \sum_{k=1}^D M_k^\dagger M_k = 1_S \]
	We already know how to go from Stinespring to Kraus, we now show the other direction.%SHOW ANYWAY
	
	\subsubsection{From Kraus to Stinespring}
	Given a Kraus set $\{M_k\}_{k=1,\dots,D}$, we want to define a system $E$, a unitary $U_{SE}$ and a state $\ket{0}$ s.t.
	\[ \Phi\left[\hat{\Theta}\right] = \Tr_E\left[ U_{SE}(t) \left(\hat{\Theta}\otimes \ket{0}_{E}\bra{0} \right) U^\dagger_{SE}(t) \right] \]
	Start with $E$ s.t. dim$(\HH_E) = D+1$ (actually, $D$ is enough, but the explicit construction is more difficult). Let $\ket{0}$ be any state of $E$ and extend it to an orthonormal basis $\{\ket{k}_{k=0,\dots,D}\}$ of $\HH_E$. Let the evolution be defined by the hamiltonian
	\[ H_{SE} = \omega \sum_{K=1}^{D} M_k\otimes \ket{k}_E\bra{0} + h.c. \]
	Then it's easy to show that for pure states $\ket{\psi}_S$ we get
	\[ \ket{\psi}_S\ket{0}_E \rightarrow \cos(\omega t) \ket{\psi}_S\ket{0}_E - i \sin(\omega t) \sum_{k=1}^{D} M_k \ket{\psi}_S \otimes \ket{k}_E \]
	For $\omega t= \dfrac{\pi}{2}$ we have
	\[ \ket{\psi}_S\bra{\psi} \rightarrow \sum_{k=1}^{D} M_k \ket{\psi}_S\bra{\psi} M_k^\dagger \]
	The extension to the general density matrix case follows from linearity.
	
	\subsubsection{Axiomatic representation}
	We can describe the set of all quantum channels as the set of
	\begin{itemize}
		\item Linear
		\item Completely positive
		\item Trace preserving
	\end{itemize}
	maps. We remind the reader of the definition of complete positivity. Given $\Phi$ superoperator on $S$, $\Phi$ is completely positive if for all ancillas $A$
	\[ (\Phi\otimes 1_A):\, \mathcal{L}(\HH_S\otimes \HH_A) \rightarrow \mathcal{L}(\HH_S\otimes \HH_A) \]
	is completely positive. Complete positivity implies positivity, but not vice-versa. %GIVE EXAMPLE
	This stronger requirement is necessary when we remember that we can always view any system $S$ as part of a larger system $S+A$, where $A$ maybe is far away and irrelevant.
	
	It is easy to show that all representations fulfill these axioms. We now exhibit an explicit Kraus set for a given channel $\Phi$ satisfying the axioms. To do so, we introduce the Choi-Jamiolkowski isomorphism.
	
	\subsection{Choi-Jamiolkowski construction}
	Given $\Phi$ acting on $S$, we construct an ancilla $A$ with the same dimensionality $d$ (assumed finite, but extensions do exist).
	We first consider any orthonormal basis $\ket{k}_S$, $\ket{k}_A$ and construct
	\[ \ket{\psi_M}_{SA} = \sum_{k=1}^{d} \dfrac{\ket{k}_S\otimes\ket{k}_A}{\sqrt{d}} \]
	maximally entangled state. The Choi state is then defined as
	\[ \rho_{CJ}^\Phi = \left(\Phi\otimes 1_A\right) (\ket{\psi_M}_{SA}\bra{\psi_M}) \]
	Notice that it is indeed a state. The choice is not unique because we can always change basis. Nevertheless we can now write down a formula (proof by substitutions) for the evolution of any pure state of $S$ in terms of $\rho_{CJ}^\Phi$
	\[ \ket{\psi}_S\bra{\psi} \rightarrow \Phi\left(\ket{\psi}_S\bra{\psi}\right) = d\, \prescript{}{A}{\braket{\psi^* | \rho_{SA}^\Phi | \psi^*}}_A \]
	where $\ket{\psi^*}_A=\sum \alpha_k^* \ket{k}_A$ if $\ket{\psi}_S=\sum \alpha_k \ket{k}_S$. We can rewrite it as
	\begin{equation}
	\label{eq:1}
	\Phi\left[ \rho_S \right] = d\, \Tr_A\left[ \left(1_S \otimes \rho_A^t \right) \rho_{SA}^\Phi \right]
	\end{equation}
	We immediately see the equivalence when $\rho_S$ is pure (note that $\rho_A^t=\rho_A^*$ due to hermiticity. Note the dependence on the basis chosen to define $\rho_{CJ}$); the extension to generic density matrix $\rho_S$ follows from linearity.
	
	To sum up, $\rho_{CJ}^\Phi$ containts all the information on $\Phi$.
	
	\subsubsection{From axiomatic to Kraus representation}
	Given $\Phi$, construct the Choi state $\rho_{CJ}^\Phi$ and put it in spectral form
	\[ \rho_{CJ}^\Phi = \sum_{\kappa=1}^{d^2} \lambda_\kappa \ket{\kappa}_{SA}\bra{\kappa} \]
	where $\lambda\ge 0$, $\sum \lambda_\kappa = 1$. Substituting in \ref{eq:1} we obtain
	\[ \Phi\left( \ket{\psi}_S\bra{\psi} \right) = d\, \sum \lambda_\kappa \prescript{}{A}{\braket{\psi^*|\kappa}}_{SA} \braket{\kappa|\psi^*}_A \]
	Defining
	\[ \ket{\chi_{l \kappa}}_S \equiv \prescript{}{A}{\braket{l|\kappa}}_{SA} \]
	we get
	\begin{equation}
	\label{eq:2}
	\Phi\left( \ket{\psi}_S\bra{\psi} \right) = d\, \sum \lambda_\kappa \alpha_l\alpha^*_m \ket{\chi_{l\kappa}}_S\bra{\chi_{m\kappa}}
	\end{equation}
	which is \emph{almost} of the form $\sum M_k \ket{\psi}\bra{\psi}M_k^\dagger$. To get there, we further define
	\[ M_\kappa \ket{l}_S  = \sqrt{d\lambda\kappa} \ket{\chi_{l\kappa}}_S \]
	so that from \ref{eq:2} we get
	\[ \Phi\left( \ket{\psi}_S\bra{\psi} \right) = \sum \alpha_l\alpha^*_m M_\kappa \ket{l}_S\bra{m}  M_\kappa^\dagger \]
	Summing over $l,m$ finally results in
	\[ \Phi\left( \ket{\psi}_S\bra{\psi} \right) = \sum_{\kappa=1}^{d^2} M_\kappa \ket{\psi}_S\bra{\psi} M_\kappa^\dagger \]
	The identity
	\[ \Tr_S\left[\Phi\left(\ket{\psi}_S\bra{\psi}\right)\right] = 1 = \sum_\kappa \braket{\psi|M_\kappa^\dagger M_\kappa | \psi} \]
	holds for all $\ket{\psi}$ and shows that $\sum M_\kappa^\dagger M_\kappa = 1$, implying that $\{M_\kappa\}$ is a good Kraus set. Our proof also shows that you can always do with at most $d^2$ Kraus operators, where $d$ is the dimensionality of your system $S$.
	
	\section{Lecture 2}
	We begin with 2 two observations.
	\paragraph{Upper bounds}
	We observe that unitary evolution $\rho\rightarrow U\rho U^\dagger$ is a CPT map whose Kraus set is singlet $\{U\}$, showing that $d^2$ is \emph{an upper bound} on the \emph{needed} Kraus operators.

	\noindent Further, considering that a Kraus representation $\{M_k\}_{k=1,\dots,D}$ can be converted to a Stinespring representation where the environment has dimension $D$, we deduce that a system of dimensionality $d$ always admits a Stinespring representation with environment at most $d^2$ dimensional.
	\vspace{2mm}
	
	\paragraph{Process tomography}
	In analogy with state tomography, we want an experimental procedure which will allow us to determine the action of some unknown map $\Phi$ with finitely many measurements (to arbitrary accuracy). There are two ways to go.
	\begin{enumerate}
		\item We can appropriately choose $d^2$ states $\{\rho_j\}$, let $\Phi$ act on each and do state tomography. Notice that if we need $n$ copies of a state to determine it with the prescribed accuracy, then $n*d^2$ state preparations (and measurements) will be required.
		\item We can be clever and exploit the Choi state. First, notice that you can prepare it experimentally: you attach an ancilla $A$ to $S$, prepare $\ket{\psi_M}_{SA}$ and let $\Phi$ act on $S$. Then, doing state tomography $n$ times, you determine $\rho_{CJ}^\Phi$ and therefore $\Phi$.
	\end{enumerate}
	The speedup of the second method relies on our ability to construct a maximally entangled state. This is a first instance of \emph{entanglement as a resource}.
	
	\subsection{Distances on the set of quantum channels}
	We begin with a few remarks on the set $\mathcal{C}(\HH_S)$ of quantum channels on $S$.
	\begin{enumerate}
		\item $\mathcal{C}$ is closed under convex combinations
		\item $\mathcal{C}$ is closed under composition (which is \emph{not} commutative)
		\item Composition gives $\mathcal{C}$ a semigroup structure (remember that noisy channels have no inverse)
	\end{enumerate}
	\vspace{2mm}
	\subsubsection{Trace distance}
	We now introduce our first distance on $\mathcal{C}(\HH_S)$. The idea is to apply $\Phi_{1,2}$ to a state and measure their distance, than take the $\sup$ over all states.
	\[ D_1(\Phi_1, \Phi_2) = \sup_{\rho\in \sigma(\HH_S)} d_1\left( \Phi_1(\rho), \Phi_2(\rho) \right) \]
	where $d_1$ is the trace distance between states. This distance is not very good because entanglement with an ancilla should help distinguish quantum channels (see ref. [4-8] in \href{https://arxiv.org/abs/1004.4110}{Benenti,Strini, 2010}).
	
	\subsubsection{Diamond distance}
	We instead consider the diamond distance $D_\diamond$, defined as
	\[ D_\diamond\left(\Phi_1, \Phi_2 \right) = D_1\left( \Phi_1\otimes 1_A, \Phi_2 \otimes 1_A \right) \]
	where $A$ is an ancillary system of dimensionality equal to that of $S$.
	
	\noindent This distance is bounded even when considering multiple copies of the same channel.
	
	We notice that
	\[ D_\diamond(\Phi_1, \Phi_2) \ge d_1\left( (\Phi_1\otimes 1_A)(\rho_M), (\Phi_2\otimes 1_A)(\rho_M) \right),\quad \rho_M = \ket{\psi_M}_{SA}\bra{\psi_M} \]
	so
	\[ D_\diamond (\Phi_1, \Phi_2) \ge d_1\left(\rho_{CJ}^{\Phi_1},\,\, \rho_{CJ}^{\Phi_2}\right) \]
	
	\subsubsection{Choi state peculiarity}
	The Choi-Jamiolkowski construction is often dubbed "isomorphism" but, as we will now see, the Choi state $\rho_{CJ}$ possesses some nontrivial properties. Taking the trace over $S$ and making various trivial substitutions results in
	\[ \Tr_S\left[\rho_{CJ}\right] = \dfrac{1}{d} \sum_{j,k=1} \Tr\left[\Phi(\ket{j}_S\bra{k})\right] \ket{j}_A\bra{k} \]
	exploiting the trace preserving property of $\Phi$ we get
	\[ \Tr_S\left[\rho_{CJ}\right] = \dfrac{1}{d} \sum_k \ket{k}_A\bra{k} = \dfrac{1}{d} 1_A \]
	Therefore a Choi state reduces to a maximally mixed state on the ancilla.
	
	\subsection{Adjoint channel}
	We know from standard quantum mechanics that we can consider equivalently the Schroedinger picture or the Heisenberg picture. We already generalized the Schroedinger picture to CPT maps $\Phi$, we now want to do the same for the Heisenberg picture.
	\vspace{2mm}
	
	\noindent Working in Kraus representation, $\Phi(\rho) = \sum M_k \rho M_k^\dagger$.
	\[ \braket{O}_{\Phi(\rho)} = \Tr\left[ \hat{O}\, \Phi(\rho) \right] = \Tr\left[\sum_k \hat{O}\, M_k\rho M_k^\dagger \right] =
	\Tr\left[ \left(\sum_k M_k^\dagger \hat{O}M_k\right)\rho \right] \equiv \Tr\left[ \Phi_H \rho \right] \]
	where we defined ($H$ stands for Heisenberg)
	\[ \Phi_H(\hat{O}) = \sum_k M_k^\dagger \hat{O} M_k \]
	
	$\Phi_H$ is called the adjoint channel, because remembering that $\braket{A|B} = \Tr\left[A^\dagger B\right]$ is a scalar product, we get
	\[ \braket{A|\Phi(B)} = \braket{\Phi_H(A)|B} \]
	It is a linear, completely positive superoperator, but it is \textbf{not} trace preserving.
	\[ 1 = \sum M_k^\dagger M_k \neq \sum M_k M_k^\dagger \]
	\noindent Indeed this property is replaced by $\Phi_H(1) = 1$, which is easily verified and goes under the name of \emph{unital property}.
	\vspace{3mm}
	
	The action of $\Phi_H$ in Stinespring representation is left as an exercise, but (my) result is
	\[ \hat{O}_S \rightarrow \prescript{}{E}{\braket{0|U_{SE}^\dagger (\hat{O}_S\otimes 1_E) U_{SE}|0}_E} \]
	
	\subsection{Complementary channel}
	Consider the Stinespring representation of a quantum channel $\Phi$. We can picture the evolution as a sort of "scattering" of system $S$ in state $\rho$ with system $E$ in state $\ket{0}_E$. The output is of course
	\[ U_{SE} \left(\rho\otimes \ket{0}_E\bra{0}\right) U_{SE}^\dagger \]
	If we trace out $E$ we are left with $\Phi(\rho)$ by definition, but if we trace out $S$ then we are left with a state of $E$. We can therefore define
	\[ \tilde{\Phi}: \sigma(\HH_S)\rightarrow \sigma(\HH_E),\qquad \tilde{\Phi}(\rho) = \Tr_S\left[ U_{SE} \left(\rho\otimes \ket{0}_E\bra{0}\right) U_{SE}^\dagger \right] \]
	This map is easily verified to be linear completely positive and trace preserving. Note that complete positivity means that a positive operator, which (up to rescaling by a positive constant) we can assume to be a density matrix $\rho_{SA}$, gets mapped to a positive operator $\rho'_{AE}$.
	
	There is some freedom in choosing the Stinespring representation, but two different choices leading say to $\tilde{\Phi}$, $\tilde{\Phi}'$ are unitary equivalent, meaning there exist $U,V$ unitary s.t. $\tilde{\Phi}'(\rho) = V \tilde{\Phi}\left( U \rho U^\dagger \right) V^\dagger$ (proof left as an exercise).
	
	Lastly, notice that if we work in physical representation instead of Stinespring then our freedom in choosing the representation is even larger.
	We will end up with $\tilde{\Phi}_W$, $\tilde{\Phi}_W'$ ($_W$ stands for \emph{weak}) not necessarily unitary equivalent. The set of weak complementary channels originating from a given $\Phi$ is considerably harder to study, because different $\tilde{\Phi}$'s represent different physics.
	
	\section{Lecture 3}
	We begin with a comment on complementary channels from last lecture. Knowing $\Phi(\rho),\tilde{\Phi}(\rho)$ means knowing the two partial traces of the full system $SE$, but we are still missing correlations. Conclusion: $\Phi,\tilde{\Phi}$ contain less information than $U_{SE}$.
	
	\subsection{Degradability}
	We say that a channel $\Phi$ is degradable if there exists $\Lambda:\mathcal{L}(\HH_S)\rightarrow\mathcal{L}(\HH_E)$ LCPT s.t.
	\[ \tilde{\Phi}(\rho) = \Lambda \circ \Phi(\rho) \quad \forall \rho \]
	The physical meaning is that we can \emph{degrade} the state of the system $S$ through some noise and obtain the state of $E$. The state of the environment contains no additional information; $\Phi$ can carry more information than $\tilde{\Phi}$.
	
	\paragraph{Example} $\Phi = 1_S$ is a degradable channel. Proof: Given $\tilde{\Phi}$ complementary channel of the identity, I claim that $\Lambda = \tilde{\Phi}$ is the right map. It is LCPT and $\tilde{\Phi} = \tilde{\Phi} \circ 1$.
	
	\subsection{Anti-degradability}
	We say that a channel $\Phi$ is anti-degradable if there exists $\Lambda':\mathcal{L}(\HH_E)\rightarrow\mathcal{L}(\HH_S)$ LCPT s.t.
	\[ {\Phi}(\rho) = \Lambda' \circ \tilde{\Phi}(\rho) \quad \forall \rho \]
	Remarks opposite to the previous case hold.
	
	\paragraph{Example} $\Phi(\rho) = \bar{\rho}$. Turn this into a well defined quantum channel (remember the domain is $\mathcal{L}(\HH_S)$, not $\sigma(\HH_S)$), prove LCPT and show that it is anti-degradable.
	
	\emph{Solution} We begin by defining $\Phi(\theta) = \Tr[\theta] \bar{\rho}$. It reduces to the right $\Phi$ on density matrices and it is trace preserving by definition. Linearity being obvious, we are left with complete positivity. Instead of proving it, we take a different route. We can exhibit a Stinespring representation of $\Phi$, that is a unitary $U_{SE}$ s.t., tracing out $E$, $\rho\rightarrow \bar{\rho}$. Then we know that a superoperator $\Phi$ given in Stinespring representation is LCPT.
	
	Write $\bar{\rho} = \sum_{k=1}^{d} \lambda_k \ket{k}_S\bra{k}$, $\lambda_k\ge 0$, $\braket{j|k} = \delta_{jk}$ a basis. We will define $d$ unitary matrices in a moment, but for now let us call them $V_i$. We define $U_{SE}$ on $\ket{k}_S\ket{0}_E$ as
	\[ U_{SE} (\ket{i}_S\ket{0}_E) = \sum_{k=1}^{d} \sqrt{\lambda_k} \ket{k}_S \otimes (V_i\ket{k}_E) \]
	The idea was to write the most general purification of $\bar{\rho}$, where we keep the dependence of $V$ on the input $i$. A unitary extension of $U_{SE}$ to all $\HH_S\otimes \HH_E$ is always possible if we prove
	\[ \prescript{}{E}{\bra{0}} \prescript{}{S}{\bra{i}} U_{SE}^\dagger U_{SE} \ket{j}_S\ket{0}_E = \delta_{ij} \]
	After a few steps we are left with
	\[ \sum_{k} \lambda_k\prescript{}{E}{\braket{k|V_i^\dagger V_j | k}_E} = \delta_{ij} \]
	All is left to do is pick the $d$ unitaries s.t. $\prescript{}{E}{\braket{k|V_i^\dagger V_j | k}_E} = 0$ if $i\neq j$ ($i=j$ is trivial). Observe that
	\[ V_k \ket{i} = \ket{i+k-1 \mod d},\qquad V_k^\dagger\ket{i} = \ket{i+1-k \mod d} \]
	is unitary and satisfies the above requirement.
	
	Finally, $\Phi$ is anti-degradable because $\Phi = \Lambda' \circ \tilde{\Phi}$ holds if we simply pick $\Lambda = \Phi$, which is LCPT as we have just seen.
	
	\subsection{Weak (anti-)degradability}
	Non mi è chiaro un punto: le definizioni ok. Se però mi dici weak anti-deg coimplica anti-deg, non mi è chiaro se nel trasformare da $\ket{0}_E$ in $\tau_E$ devo mantenere fisso l'output su $S$ o l'unitaria $U_{SE}$.
	
\end{document}